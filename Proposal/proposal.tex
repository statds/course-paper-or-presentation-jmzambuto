\documentclass[12pt]{article}
\usepackage{amsmath}
\usepackage[margin=1 in]{geometry}
\usepackage{graphicx}
\usepackage{booktabs}
\usepackage{natbib}
\usepackage{lipsum}
\usepackage[colorlinks=true, citecolor=blue]{hyperref}

\title{STAT 3494W Proposal}
\author{Jessica Zambuto\\
Department of Statistics, University of Connecticut} 

\begin{document}
\maketitle

\section{Introduction}
\label{sec:intro}
I will be researching the effects of the Covid-19 pandemic on influenza vaccination rates within the United States. This topic
is important because research has shown that "Vaccinated patients testing positive for COVID-19 were less likely to require hospitalization 
(odds ratio, 0.58, 95\% CI 0.46-0.73; $P< .001$), or mechanical ventilation (odds ratio, 0.45, 95\% CI 0.27-0.78; P = .004) and had a shorter 
hospital length of stay (risk ratio, 0.76, 95\% CI 0.65-0.89; $P< .001$)" \citep{conlon2021impact}. Thus, by determining if there are decreases
in vaccination rates, medical and public health professionals may be able to implement strategies for increasing the rate, so that there is less 
severity in covid infections.  

\section{Specific Aims}
\label{sec:aims}
I hypothesize that the influenza rates among children ages 6+ months-17 years, adults ages 18-64 years, and the elderly ages 65+ increased during 
the pandemic (2020-2021) and will continue to after (2022). Research has shown that there is a statistically signfificant increase in vaccination 
rates from before the pandemic to during among healthcare and non-healthcare workers \citep{conlon2021impact}. Additionally, in a study conducted 
in Italy, the majority interviewed agreed that vaccines are crucial to public health and should be mandatory. These attitudes can influence whether
a person recieves the influenza vaccine and because this study indicates that people have a positive view on it, they would be more likely to 
get it during the pandemic \citep{domnich2020attitudes}. Additionally, in another study, " 21\% of parents whose children did not receive the 2019-2020 
influenza vaccine reported that the COVID-19 pandemic made them more likely to have their child receive the 2020-2021 influenza vaccine" \citep{sokol2020covid}.


\section{Data Description}
\label{sec:data}
The data I will be using for this research project comes from the Center for Disease Control and Prevention Website called, 
"Influenza Vaccination Coverage for All Ages (6+ Months)". The dataset was collected using the National Immunization Survey-Flu and 
Behavioral Risk Factor Surveillance System at the national, regional, and state levels by age and race. The dataset is updated monthly and 
includes the name of the influenza vaccine distributed, the state, the year, the age group, the percent estimate of those vaccinated, 
the 95\% confidence interval for the estimate, and the sample size. I am interested in analyzing the data at a race, age, and year 
level. This dataset contains 169,293 observations. I will also be using the CDC's, "Vaccine Coverage Among Health Care Personnel" dataset, 
which contains the same variables as the first dataset mentioned except for age and is only based on healthcare workers. The data was collected 
using the influenza vaccination performance measurement from the National Healthcare Safety Network. I am interested in analyzing this dataset 
on a yearly level. This dataset contains 1456 observations. 

\section{Research Design and Methods}
\label{sec:design}
To begin my research, I will first compile a literature review to have a better understanding of existing results. I plan on only analyzing the, 
"Influenza Vaccination Coverage for All Ages" dataset on a year, race, and age basis and the, "Vaccine Coverage Among Health Care Personnel" 
dataset on a year basis. However, these variables of interest are subject to change if I find confounding variables or others that
are associated with recieving the influenza vaccination based on my literature review. I will analyze the two datasets by using the two sample z 
test for proportions. I will first compare the proportions of the general population who recieved the influenza vaccine prior to the COVID-19 
pandemic defined as 2018-2019 and during the pandemic defined as 2020-2021. I will then compare the proportions of the general population who
recieved the influenza vaccine during the pandemic (2020-2021) to those who did in 2022. I will repeat these steps for the, "Vaccine Coverage Among 
Health Care Personnel" dataset. The steps above will be repeated for the "Influenza Vaccination Coverage for All Ages" dataset, but this time not only
comparing before, during, and after the pandemic, but different races, so the z test will be conducted two times for each race (white, black, asian,
hispanic, and other). The Z test will again be repeated for before, during, and after the pandemic, but for different age groups defined by 6 months-17 years, 
18-49 years, 50-64 years, and greater than 65 years of age. All of these tests will allow me to determine if there is statistically signficant evidence in favor
of in an increase in influenza vaccination rates during and after the Covid-19 pandemic and if race has an association among these results.

\section{Discussion}
\label{sec:discussion}
As stated in the specfic aims section of this proposal, I expect to find an increase in influenza vaccination rates among each age group during and after the 
pandemic. However, I do expect there to be a difference in vaccination rates among each racial group. This is because one study found that patients who recieved 
the vaccine tended to be Caucasian and  the multivariable logistic regression model found that significant independent predictors
of influenza vaccination included race \citep{conlon2021impact}. Similarily, another study found that "more U.S.-born than foreign-born individuals and more 
non-Hispanic Whites than non-Hispanic Blacks and Hispanics took the flu vaccine. Considering nativity and Hispanic ethnicity, foreign-born Hispanics were less 
likely to take the flu vaccine than their U.S.-born and non-Hispanic White counterparts" \citep{jang2021factors}. This is most likely due to health care access 
disparties, language barriers, and hesitancy. 

\section{Conclustion}\
\label{sec:Conclusion}
Research has shown that recieving the influenza vaccine may be associated with less severe covid-19 cases. Thus, it is important to have high influenza vaccination
rates, but this is only possible if there is equal acccess to health care among all racial groups regardless of socioeconomic status. Currently, this is not the case
within the United States. By studying how the influenza vaccination rates have changed before, during, and after the pandemic among different age groups, race, and
healthcare versus non-healthcare workers, I hope to have a better understanding of how each group has been effected by the pandemic. If there are dispareties among the 
racial groups, this information can be used amomng researchers to implement new polices or strategies, so that these groups can have equal access to the vaccine. 

\bibliography{refs}
\bibliographystyle{chicago}
\end{document}
